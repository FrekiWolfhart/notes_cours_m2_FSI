\section{Cours 2}
Le plus grand probl\`eme des protocoles cryptographiques n'est pas le chiffrement, mais le partage s\'ecuris\'e des
cl\'es.

On peux faire un syst\`eme similaire au VPN sans protocoles cryptographiques, en utilisant un syst\`eme filiaire.

L'envoi de mails via le protocole SMTP peux utiliser le protocole STARTSSL.

Il y a deux moyens de s\'ecuriser des pacquets web:
\begin{itemize}
  \item AH: on peux lire l'information, mais on ne peux pas la modifier de l'ext\'erieur.
  \item ESP: one ne peux ni lire ni modifier l'information de l'ext\'erieur.
\end{itemize}
Les en-t\^etes des deux protocoles sont similaires.

Les RFC sont des protocoles internet standardis\'es.

Afin de contrer les attaques sur l'int\'egrit\'e de la session, on peux hacher et signer les pr\'ec\'edents messages et
les renvoyer vers le serveur, afin qu'il puisse v\'erifier que le contenu n'ai pas \'et\'e alt\'er\'e.

La compression est suppos\'ee diminuer les r\'ep\'etitions, afin d'\'eviter les attaques l'utilisant.
En plus, cela diminue la taille du pacquet.
Mais, la taille du message compress\'e peux donner certaines informations sur les octets et bits du message initial.

DHE: Deffie-Helmann Ephemeral
Ephemeral: les cl\'es sont jet\'ees d\`es qu'elles ne sont plus utiles.

UDP a une version de TLS qui est adapt\'ee afin de g\'erer le probl\`eme de pacquet qui se perd, et qui s'appelle DTLS.
