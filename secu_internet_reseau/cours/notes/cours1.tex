\section{Cours 1}
\hspace*{-3cm}\begin{tabular}{|c|c|c|c|}
\hline
\multicolumn{2}{|c|}{ } & cryptographie sym\'etrique & cryptographie asym\'etrique\\
\hline
\multicolumn{2}{|c|}{confidentialit\'e} & chiffrement & chiffrement \`a clef publique\\
\hline
\multicolumn{2}{|c|}{int\'egrit\'e} & \multirow{2}{3.5cm}{code d'authentification de message MAC} &
\multirow{4}{2cm}{signature num\'erique}\\
\cline{1-2}
\multirow{2}{4cm}{authentification} & donn\'ees & & \\
\cline{2-2}
 & entit\'es & & \\
\cline{1-3}
\multicolumn{2}{|c|}{non r\'epudation} & aucune primitive & \\
\hline
\end{tabular}
Il n'y a aucune primitive de non r\'epudation pour la cryptographie asym\'etrique car il n'est pas possible de
diff\'erencier les interlocuteurs.
\medskip
Il existe trois types d'architectures:
\begin{enumerate}
  \item ferm\'ee: Il y a peu d'acteurs, ou la structure est autonome
  \item hi\'erarchis\'ee: Il y a de nombreux acteurs et peu d'autorit\'es de confiance
  \item d\'ecentrali\'ee: Il y a de nombreux acteurs et la certification des connaissances se base se fait de
  proche en proche
\end{enumerate}

Chaque type d'architecture a \c ca propre fa\c con d'\^etre s\'ecuris\'ee:
\begin{enumerate}
  \item ferm\'ee: Kerberos, IPSec
  \item hi\'erarchis\'ee: PKI \`a base d'authentification X509: IPSec, SSL
  \item d\'ecentralis\'ee: PGP/GnuPG, r\'eseau P2P/F2F crypt\'e, \dots
\end{enumerate}
\medskip
La clef doit \^etre sign\'ee par une figure d'autorit\'e, qui peux \^etre soi-m\^eme. Un clef sign\'ee par soi-m\^eme
est dite auto-sign\'ee.
\smallskip
PKI= Public Key Infrastructure
\smallskip
Il est possible de r\'evoquer un certificat si les informations ont chag\'ees, ou bien si la clef priv\'ee est
compromise, ou encore si il y a des changements dans l'institution.

Certaines fonctions de hachage sont d\'econseill\'ees, d\^u aux failles et aux collisions, comme MD5.
