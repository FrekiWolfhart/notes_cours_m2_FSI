\section{Introduction}
La s\'ecurit\'e n'est pas seulement un probl\`eme technique. En effet, si une cl\'e priv\'ee devient publique, alors
peu importe l'algorithme de chiffrage, il ne sera plus aussi r\'esistant car on pourra utiliser la cl\'e pour
d\'echiffrer les messages chiffr\'es, ou pour usurper l'identit\'e de l'auteur.

La s\'ecurisation des terminaux est toute aussi importante que la s\'ecurisation des terminaux, afin d'\'eviter:
\begin{itemize}
  \item les sniffeurd de claviers et chevaux de troie
  \item la collecte de donn\'ees sans accord
  \item les techniques TEMPEST, qui peuvent entra\^iner la collecte d'infos et de daat sans programmes externes
  \item le vol et/ou la perte de diff\'erents moyens de stockage:
  \begin{itemize}
    \item disque dur
    \item cl\'e USB
    \item t\'el\'ephone
    \item ets\dots
  \end{itemize}
  \item les techniques d'ing\'egnerie sociale
  \item etc\dots
\end{itemize}

Il existe deux types de menaces:
\begin{itemize}
  \item les menaces involontaires, qui sont li\'ees \`a des accidents ou des erreurs, et qui sont li\'ees \`a la
  fiabilit\'e et la qualit\'e
  \item les menaces volontaires, qui sont li\'es \`a des attaques, et qui sont li\'ees \`a la s\'ecurit\'e
\end{itemize}

Lors du design de syst\`eme d'informations, on peux consid\'erer un accident tr\`s improbable comme inexistant.

Il y a deux types d'attaques de r\'eseaux:
\begin{itemize}
  \item l'attaque passive, ou d'\'ecoute, qui consiste en un acc\`es de contenu dont on est pas destinataire ou
  propri\'etaire, dont les sources peuvent \^etre:
  \begin{itemize}
    \item TCP/IP non s\'ecuris\'e, ce qui m\`ene \`a une vue int\'egrale des pacquets, ce qui permet une attaque du
    protocole afin de devenir interm\'ediaire
    \item un document publi\'e par erreur, qui peux \^etre trouv\'e sur internet
    \item une violation des syst\`emes de contr\^ole d'acc\`es, qui peux \^etre d\^u \`a des mot de apsse faibles ou des
    injections SQL
    \item des \'ecoutes l\'egales
  \end{itemize}
  \item l'attaque active, qui consiste en une interception de contenu dont on est ni le propri\'etaire, ni le
  destinataire, et peux permettre la modification des donn\'ees en plus de l'acc\`es, et dont la source peut \^etre:
  \begin{itemize}
    \item TCP/IP non s\'ecuris\'e, comme dans le point pr\'ec\'edent
    \item une attaque sur des protocoles permettant le vol de la session
  \end{itemize}
\end{itemize}

On peux se prot\'eger des attaques sur le r\'eseaux en appliquant de bonnes pratiques:
\begin{itemize}
  \item En int\'egrant la s\'ecurit\'e dans la conception:
  \begin{itemize}
    \item en v\'erifiants les besoins de chaque utilisateurs
    \item en ne donnant pas aux utilisateurs plus de droits que requis pour leur travail
  \end{itemize}
  \item En ayant une bonne admininstration:
  \begin{itemize}
    \item les mises \`a jour permettent la correction d'erreurs et de bugs dans le code
    \item le contr\^ole des acc\`es permet une meilleure gestion des droits serveurs et machines
    \item la surveillance permet la v\'erification du syst\`eme et des utilisateurs
    \item les audits permettent de tester le niveau de s\'ecurit\'e
  \end{itemize}
\end{itemize}

Il y a diff\'erentes m\'ethodes de cryptage de donn\'ees, qui ont leur propres vuln\'erabilit\'es:
\begin{itemize}
  \item m\'ethode par substitution, qui est vuln\'erable \`a une attaque par fr\'equence, qui consiste \`a analyser la
  fr\'equences des lettres pour d\'echiffrer le message
  \item les m\'ethodes complexes, et param\'etr\'ees par cl\'es, qui sont solides
\end{itemize}
