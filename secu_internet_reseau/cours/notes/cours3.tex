\section{Cours 3}
On peux tromper les pare-feu via des tunnels, en envoyant des donn\'ees par un canal diff\'erent.

On pr\'ef\`ere les r\`egles sans-\'etats, car moins co\^uteuses, mais certains protocoles obligent celles avec \'etats.
Le pare-feu aide donc \`a d\'eterminer ce qui est accessible depuis l'ext\'erieur.

Une r\`egle a deux aprties:
\begin{enumerate}
  \item garde: condition d\'eterminant l'action \`a r\'ealiser
  \item action: l'action \`a r\'ealiser
\end{enumerate}

Sur iptables, si la r\`egle est un LOG, le logiciel continue de chercher des r\`egkes apr\`es.
