\section{TD 2}
\subsection*{Question 1}
C'est un document v\'erifiant qu'une clef publique appartient bel et bien \`a un utilisateur donn\'e.\\
Le principe de v\'erification de certificat consiste \`a faire v\'erifier la validit\'e du certificat par une
autorit\'e reconnue, en commen\c cant par v\'erifier la signature du certificat.\\
Les primitives cryptographiques utilis\'ees sont les fonctions de hachage et le chiffrement par clef publique.
\subsection*{Question 2}
\subsubsection*{Jeu A}
Le certificat A-2, qui est celui contenant la clef d'empreinte, est un certificat auto-sign\'e. Il a l'autorisation
de signer des certificat, d\^u \`a l'argument CA:true. Cependant, on ne consid\`ere comme autorit\'e que le
certificat A-1, et, A-2 n'\'etant pas sign\'e par A-1, on ne peux pas accepter la clef d'empreinte comme
appartenant \`a l'utilisateur.
\subsubsection*{Jeu B}
Le certificat B-2 est sign\'e par un certificat inconnu. On ne peux donc pas accepter la clef d'empreinte comme
appartenant \`a l'utilisateur.
