\section{Cours 2}
Un jeu de parit\'e est un ensemble G=(V,E,$\Omega$), avec:
\begin{itemize}
  \item V, des positions
  \item E, des arr\^etes
  \item $\Omega$, une fonction
\end{itemize}
Les positions du jeu de parit\'e peuvent apprtenir \`a un des deux joueurs.
Joueur 1(aussi appel\'e Eve):
\begin{itemize}
  \item Poss\`ede les positions circulaires
  \item Gagne si la plus grande priorit\'e vue infiniment souvent est paire
\end{itemize}
Joueur 2(aussi appel\'e Adam):
\begin{itemize}
  \item Poss\`ede les position carr\'ees
  \item Gagne si la plus grande priorit\'e vue infiniment souvent est impaire
\end{itemize}
Une strat\'egie $\sigma$ est une fonction qui regarde l'historique du jeu. Elle est dite positionnelle si elle ne
regarde pas le dit historique.\\
Dans la partie $\pi$, si $\pi_1\in$V$_2$, alors $\pi_{i+1}$=$\sigma$($\pi$[ \,0,i] \,).\\
\medskip
\textbf{Th\'eor\`eme:}Si un joueur gagne dans G, il gagne avec une strat\'egie positionnelle.\\
\medskip
Algortihme de r\'esolution des jeu de parit\'e:\\
Compl\'exit\'e $\Theta$(n$^d$):\\
\begin{algorithm}
  \caption{Algorithme de Lielonka}
  \begin{algorithmic}
  \STATE f$_1$(G,d)
  \REPEAT
    \STATE N$_d=$\{v$\in$V$\mid\Omega$(v)=d\}
    \STATE G'=G$\backslash$Att$_1$(G, N$_d$)
    \STATE W$_2$=f$_2$(G, d-1)
    \STATE G=G$\backslash$Att$_2$(G,W$_2$)
  \UNTIL{W$_2$=$\varnothing$}
  \STATE return G
  \end{algorithmic}
\end{algorithm}
\medskip
Une variante des jeux de parit\'e sont les jeux \`a registres. En plus de jouer, \`a chaque tour, le joueur 1 range la
valeur sortie dans un registre. Il existe k registres.\\
Fonctionnement des registres:
\begin{itemize}
  \item mise \`a jour(le joueur 1 met \`a jour le registre i):
  \begin{itemize}
    \item j < i: r$_j$=0.
    \item r$_i$= p(la valeur sortie)
    \item j > i: r$_j$=max(r$_j$,p)
  \end{itemize}
  \item priorit\'e de sortie:
  \begin{itemize}
    \item 2i si max(r$_i$,p) est pair
    \item 2i+1 sinon
  \end{itemize}
\end{itemize}
Les conditions de victoire sont les m\^emes que pour les jeux de parit\'e.
\medskip
\textbf{Lemme:}Si le joueur 2 gagne G, il gagne G$_k$, $\forall$ k. Si le joueur 1 gagne G, $\exists$k tel que le
joueur 1 gagne G$_k$.\\
\textbf{Th\'eor\`eme:}Joueur 1 gagne G si et seulement si elle gagne G$_{log(n)+1}$. Par cons\'equent, J$_1$ gagne
G$_{log(n)+1}\Rightarrow$J$_1$ gagne G.\\
\textbf{Induction:}J$_1$ gagne G$_{log(n)+1}$ avec une strat\'egie telle que, si r$_k\geq$p est paire, alors il n'y a
pas de sortie 2k+1.\\
Cette strat\'egie ne fonctionne que si il n'y a un componant fortement connexe utilisant tout les registres. Dans ce
cas, il faut un registre de plus pour cette strat\'egie.\\
\textbf{Corollaire:} Algorithme 2$^{\Theta(log^3(n))}$: r\'esoudre G$_{log(n)+1}$ au lieu de G, et le nombre de
priorit\'es est $\Theta$(log(n)).\\
\medskip
f$_1$(G,d,p$_2$,p$_1$). Un dominion est un ensemble tel que un joueur a une strat\'egie gagnante \`a partir du dominion
et qui n'en sort pas.\\
\medskip
f$_1$ doit contenir tout les dominions J$_1$ de taille$\leq$p$_1$, et n'intersecte pas avec les dominions de J$_2$, de
taille $\leq$p$_2$.\\
f$_2$(G,d,p$_1$,p$_2$) existe aussi.\\
\begin{algorithm}
  \caption{Algorithme}
  \begin{algorithmic}
    \STATE f$_1$(G,d,p$_2$,p$_1$)
    \IF{G=$\varnothing\vee$p$\leq$1}
      \STATE return G
    \ENDIF
    \STATE G$_1$=f$_1$(G,d,p$_2$,$\lfloor\frac{p_1}{2}\rfloor$)
    \STATE Nd=\{v$\mid\Omega$(v)=d\}
    \STATE H=G$_i\backslash$Att$_1$(Nd,G)
    \STATE W$_2$=f$_2$(H,d-1,p$_1$,p$_2$)
    \STATE G$_2$=G$_1i\backslash$Att$_2$(W$_2$,G$_2$)
    \STATE G$_3$=f$_1$(G$_2$,$\lambda$,p$_2$,$\lfloor\frac{p_1}{2}\rfloor$)
    \STATE return G$_3$
  \end{algorithmic}
\end{algorithm}
