\section{Cours 3}
Les probl\`emes de synth\`ese sont une famille de probl\`emes g\'en\'eriques dont la recherche a \'et\'e initi\'ee
par Church en 1957. Cette famille introduit un champ de recherche vari\'ee, qui est partie \`a la base de la th\'eorie
algorithmique de jeux.\\
Ce sont des probl\`emes qui sont, en g\'en\'eral, difficiles ou insolubles. On peux donc se damnder quand peux-t-on
les r\'esoudre, et avec quelle compl\'exit\'e.\\
\medskip
Soit R $\subseteq$ X$\times$Y, une relation, et soit f: X$\rightarrow$Y, une fonction partielle. f uniformise
R si:
\begin{itemize}
  \item domaine(f)=domaine(R).
  \item $\forall$x$\in$domaine(R), (x,f(x))$\in$R.
\end{itemize}
On peux aussi dire que f est une fonction de choix pour R(nom moins utilis\'e).\\
\medskip
Soit S, un formalisme de relations X$\times$Y.\\
Soit P, un formalisme de fonctions de X$\rightarrow$Y.\\
\begin{algorithm}
  \caption{Probl\`eme de P-S synth\`ese}
  \begin{algorithmic}
    \REQUIRE Une relation R donn\'ee dans S(R appel\'ee sp\'ecification)
    \ENSURE Produire, pour un algo, une fonction f donn\'ee dans P(f appel\'ee programme), telle que f
    uniformise R
    \STATE R d\'esigne la paire (input,output) consid\'er\'ees comme acceptables.
    \STATE f d\'esigne le programme qui assigne \`a chaque input un output acceptable.
  \end{algorithmic}
\end{algorithm}

Pour d\'efinir des sp\'ecifications, on utilise MSO(logique pour les sp\'ecifications) sur A$\times$B.\\
Si on a u, un mot infini sur A, et v, un mot infini sur B, on note u$\otimes$v, un mot infini sur
(A,B)$^\omega$.\\
A$_a$(x)=$\bigvee_{b\in B}$(a,b)(x). B$_b$(x)=$\bigvee_{a\in A}$(a,b)(x).\\
Le formalisme utilis\'e pour les programmes est un automate fini d\'eterministe et qui, en plus de lire des
lettres, va produire des lettres.\\
Pour les sp\'ecification, on utilise la logique MSO, pour les programmes, on utilise les machines de Mealy.\\
Deux points pour un programme tr\`es simple(comme une machine de Mealy):
\begin{enumerate}
  \item R\'eactif: chaque entr\'e produit une sortie
  \item M\'emoire finie
\end{enumerate}
\medskip
Compl\'xit\'e non-\'el\'ementaire: compl\'exit\'e qui est une exponentielle infinie non terminale.\\
Complexit\'e tower: complexit\'e non-\'el\'ementaire qui poss\`ede n exponentielles, n \'etant le nombre
d\'entr\'ees.
\begin{algorithm}
  \caption{Probl\`eme de synth\`ese de Church( compl\'exit\'e tower)}
  \begin{algorithmic}
    \REQUIRE formule MSO qui d\'efinit une sp\'ecification R
    \ENSURE une machine de Mealy qui d\'efinit f, tel que f uniformise R
  \end{algorithmic}
\end{algorithm}
R\'esolution du probl\`eme de Church:
\begin{enumerate}
  \item Passer de la formule logique MSO \`a l'automate fini.
  \item Passer des automates aux jeux.
  \item Jeux de Muller.
  \item Jeux de parit\'e.
\end{enumerate}
On peux voir une relation R$\subseteq$A$^\omega\times$B$^\omega$ comme un jeu infini entre deux joueurs: J$_1$ et
J$_2$.\\
J$_1$ joue des lettres de A, J$_2$ joue des lettres de B. Ils jouent tour \`a tour et J$_1$ commence.\\
Qui gagne?
\begin{enumerate}
  \item Quand le mot ne satisfait pas la sp\'ecification.
  \item Quand le mot satisfait la sp\'ecification.
\end{enumerate}
\medskip
\textbf{Th\'eor\`eme:} Un langage L$\subseteq\Sigma^\omega$ est d\'efinissable en MSO si et seulement si il
est reconnu par un authomate de Muller/B$\ddot{u}$chi d\'eterministe.\\
Automate de B$\ddot{u}$chi: A=(Q,$\Sigma$,$\delta$,q$_0$,F):
\begin{itemize}
  \item Q: ensemble fini d'\'etats
  \item $\Sigma$: alphabet
  \item $\delta\subseteq$Q$\times\Sigma\times$Q
  \item q$_0\in$Q: \'etat initial
  \item F$\subseteq$Q ensemble des \'etats finaux
\end{itemize}
