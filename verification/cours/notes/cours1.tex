\section{Cours 1}
Le $\mu$-calculest une logique modale \`a points fixes.\\
Symboles du $\mu$-calcul:
\begin{itemize}
  \item tt: vrai
  \item ff: faux
  \item $\wedge$: conjonction
  \item $\vee$: disjonction
  \item $\square$: universel(successeur toujours vrai)
  \item $\Diamond$: existentiel(au moins un successeur vrai)
  \item $\mu$X.$\varphi$: plus petit point fixe
  \item $\nu$X.$\varphi$: plus grand point fixe
\end{itemize}
L'existentiel ne peux pas \^etre vrai si il n'existe pas de successeur \`a l'\'etat courrant.\\

Formules importantes:
\begin{itemize}
  \item $\Diamond$tt: indique la pr\'esence d'au moins un successeur
  \item $\square$ff: indique l'abscence de successeur
  \item $\Diamond$ff: comme ff, toujours faux
  \item $\square$tt: comme tt, toujours vrai
\end{itemize}

$\mu X^0.\varphi =\varphi[ \,\frac{ff}{X}] \,$.$\mu X^{0+1}.\varphi =\varphi[ \,\frac{\mu X^0.\varphi}{X}] \,$.$\lfloor\lceil\mu X.\varphi\rceil\rfloor =\bigcup_{i=\mathbb{N}}\mu X^i.\varphi$
\medskip
$$\mu X^0.a\vee\Diamond X=a\vee\Diamond ff=a$$
$$\mu X^1.a\vee\Diamond X=a\vee\Diamond a$$
$$\mu X^2.a\vee\Diamond X=a\vee\Diamond(av\Diamond a)$$
Donc, $\mu$X$^i$.a$\vee\Diamond$X $\rightarrow$ $\exists$ a dans au plus i pas.\\
Le plus petit point fixe dit donc qu'il existe un a accessible, sans pr\'eciser quand.
$$\mu X^0.a\wedge\square X=a\wedge\square ff$$
$$\mu X^1.a\wedge\square X=a\wedge(a\wedge\square ff)$$
$\mu$X$^i$.a$\wedge\square$X: a est vrai jusqu'\`a la profondeur i.\\
$\mu$X$^i$.a$\wedge\square$X=$\square$a, et tout les chemins sont finis.\\
$\nu X^0.\varphi=[ \,\frac{tt}{X}] \,$.$\nu X^1.\varphi=[ \,\frac{\nu X^0\varphi}{X}] \,$.$\lfloor\lceil\nu
X.\varphi\rceil\rfloor=\bigcap_{i\in\mathbb{N}}\nu X^i.\varphi$
$$\varphi=\nu X;a\wedge\Diamond X$$
$$\varphi^0=a\wedge\Diamond tt$$
$$\varphi^1=a\wedge\Diamond(a\wedge\Diamond tt)$$
$$\varphi^2=a\wedge\Diamond(a\wedge\Diamond(a\wedge\Diamond tt))$$
$\varphi^i$:$\exists$ un chemin de longueur i + 1, ayant un successeur, $\wedge\square$a.
J'ai un chemn infini avec des a partout.
\medskip
$$\varphi=\nu X;a\wedge\square X$$
$$\varphi^0=a\wedge tt$$
$$\varphi^1= a\wedge\square(a\square tt)$$
$\varphi^i$: a est vrai pendant i pas.
Donc, $\varphi$: a est toujours vrai.
\medskip
Quelques formules:
\begin{itemize}
  \item $\mu$X.$\square$X: indique que tout chemin se termine, car la formule finira toujours par $\square$ff, qui est
  l'abscence de successeur
  \item $\mu$X.($\Diamond$X)$\vee$($\nu$Ya$\wedge$.$\square$Y):Il existe une partie accessible o\`u a est toujours
  vrai.
  \item $\nu$Y.($\square$Y)$\wedge$($\mu$X.a$\vee\Diamond$X): Il existe toujours un a accessible.
  \item
  $\nu$Y.($\square$Y)$\wedge$($\neg$req$\vee$($\mu$X.ack$\vee$($\square$X$\wedge\Diamond$X))):
  si on voit un req, il y aura toujours un ack dans le futur.
  \item $\nu$X.$\mu$Y.$\Diamond$Y$\vee$(a$\wedge\Diamond$X): $\exists$ un chemin avec a infiniment souvent
  pr\'esent.
  \item $\mu$X.$\nu$Y.($\square$Y$\wedge\neg$a)$\vee$a$\wedge\square$X: $\exists$ un nombre fini de a pour toute
  ex\'ecution.
\end{itemize}
\medskip
Un jeu de parit\'e r\'eponds \`a la question: Si j'ai un graphe, est-ce que la formule est vraie dedans?\\
Un jeu de parit\'e peux cr\'eer deux types de cycles avec les points forts, qui sont nomm\'es en fonction du gagnant
lorsqu'on entre dans le cycle; Ils sont:
\begin{enumerate}
  \item cycle $\mu$-perdant
  \item cycle $\nu$-gagnant
\end{enumerate}
