\section{Cours 5}
Le service doit r\'epondre aux objectifs du client. Il est aussi important de les former afin de s'assurer que ses
fonctionnalit\'es correspondent bien \`a ce que le client souha\^ite.\\
La planification et coordination des ressources est une \'etape critique, qui permet de s'assurer que chaque changement
et mise \`a jour cause le moins de probl\`emes possibles.\\
Le plan de transition des services est une marche \`a suivre expliquant comment faire la mise en production.\\
La question de la s\'ecurit\'e doit toujours \^etre pos\'ee, m\^eme durant les mises \`a jour et changement des
services.\\
M\'ethode de gestion des changements: m\'ethode des 7R, une m\'ethode anglo-saxonne.\\
Il faut conna\^itre les relations entre les services afin d'\'eviter de casser un service en en changeant un autre.\\
La personne en charge des changements doit toujours \^etre celle qui s'occupe du changement.\\
Il faut bien pr\'eciser aux gens que la documentation ne sert pas \`a se d\'ebarasser d'eux, mais bien \`a mieux traiter les
probl\`emes.
