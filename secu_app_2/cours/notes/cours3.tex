\section{Cours 3}
Durant les op\'erations de pentesting, il est important de d\'efinir des objectifs clairs, et les moyens autoris\'es pour
y parvenir.
Si les contraintes ne sont pas respect\'ees, vous vous exposez aux sanctions d\'efinies par
\href{https://www.legifrance.gouv.fr/codes/article_lc/LEGIARTI000030939438?isSuggest=true}{l'article 323-1 du code
p\'enal}.

Le pentesting pe\^ux co\^uter cher \`a une entreprise, il faut donc que l'analyse soit la plus pouss\'ee et clair
possible, afin que cela vaille le coup pour elle. Il ne faut pas oublier d'\^etre rationnel et raisonnable en testant
les menaces, afin d\'eviter les d\'epenses superflues.

Faire des revues de code permet de trouver des probl\`emes d'impl\'ementation. Id\'ealement, cela commence d\`s que l'on
commence \`a coder, mais \c ca doit \^etre p\'ecis\'e par les demandes du projet.
L'analyse de code se doit aussi d'\^etre cibl\'ee; En effet, il est inutile de v\'erifier des probl\`emes de gestion de
m\'emoire manuelle dans un langage dans lequel la gestion de m\'emoire est automatis\'ee.
