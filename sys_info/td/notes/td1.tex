\section{\'Etude de cas 1}
Quand on analyse, on ne doit pas partir dans l'id\'ee que c'est une catastrophe. Il faut partir sur ce qui existe, et
l'am\'eliorer. Il faut aussi \'eviter d'anticiper les solutions, c'est-\`a-dire analyser avant de r\'eparer.

Forces et faiblesses du syst\`eme existant:
\begin{itemize}
  \item FORCES:
  \begin{itemize}
    \item syst\`eme de sauvegarde
    \item syst\`eme de suppervision
    \item agents identifi\'es de service d'informatique pour surveiller, et pr\'esence de DSI
    \item classement de donn\'ees critiques
  \end{itemize}
  \item FAIBLESSES:
  \begin{itemize}
    \item le mat\'eriel contenant les donn\'ees critiques \'etait hors garantie
    \item le technicien faisant le remplacement a \'et\'e mal inform\'e
    \item les sauvegardes ont \'echou\'ees depuis un moment
    \item la sauvegarde la plus r\'ecente n'\'etait pas int\`egre
  \end{itemize}
\end{itemize}

Ils pourraient r\'egler le probl\`eme en revoyant leur protocole de transmission de donn\'ees aux intervenants externes,
en documentant le mode op\'eratoire, ou bien en d\'elocalisant la gestion des serveurs.
